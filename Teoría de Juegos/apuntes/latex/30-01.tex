\Date{30 de enero de 2023}
% \DateLine{30 de enero de 2023}

\section*{Descripción de la asignatura}
% \addcontentsline{toc}{section}{Descripción de la asignatura}

\textbf{Teoría de Juegos} es un curso de herramientas esencial dentro de la serie de microeconomía de la licenciatura. Estudiaremos cómo modelar y comprender las interactuaciones estratégicas en las ciencias sociales. Estudiaremos partes de la teoría y sus aplicaciones en varios contextos. Cubriremos aplicaciones de organización industrial, de negociaciones, de reputación con interacciones repetidas, y de diseño de instituciones.

\section*{Objetivos del curso}
% \addcontentsline{toc}{section}{Objetivos del curso}

El curso tiene dos objetivos. El primero es ofrecer herramientas e ideas para aplicaciones de economía. El segundo es estimular a estudiantes que podrían dedicarse a la investigación. \textbf{El desarrollo del curso utilizará la exposición pública de ideas por parte de l@s estudiantes y la retroalimentación por mi parte. Que sus ideas sean expuestas, debatidas, y retroalimentadas públicamente es una parte crucial de su aprendizaje}.

\section*{Evaluación}
% \addcontentsline{toc}{section}{Evaluación}

La calificación del curso será el promedio de:
\begin{enumerate}
    \item Cada 2 Miércoles alternos (comenzando el 8 de Febrero) ustedes resolverán individualmente un problema que yo les proponga en clase durante media hora. Después yo resolveré ese problema y ustedes usarán mi resolución para autoevaluarse. Yo podré ajustar sus evaluaciones si claramente veo que son distintas a la solución del problema. Tendremos 9 de estos ejercicios durante el curso (\textbf{45\% en total}).
    \item Dos exámenes parciales, 20\% cada parcial (\textbf{40\% en total}).
    \item ``\textit{Quizzes}" propuestos por su laboratorista (\textbf{15\% en total}).
\end{enumerate}

\section*{Bibliografía}
% \addcontentsline{toc}{section}{Bibliografía por temas}

(W): Joel Watson (2008): \textit{An Introduction to Game Theory}, W. W. Norton \& Company

(O): Martin J. Osborne (2003): \textit{An Introduction to Game Theory}, Oxford University Press. \\

\begin{enumerate}
    \item \textbf{Ejemplos e Introducción} \\
          \textit{(O): Introducción.}
    \item \textbf{Juegos en Forma Extensiva} \\
          \textit{(W): Capítulos 2 y 9; (O): Capítulo 5.}
    \item \textbf{Juegos en Forma Normal} \\
          \textit{(W): Capítulos 3, 6, 7; (O): Capítulos 2, 3, 12.}
    \item \textbf{Equilibrio de Nash en Estrategias Mixtas} \\
          \textit{(W): Capítulos 4, 11; (O): Capítulo 4.}
    \item \textbf{Comportamiento en Contextos Dinámicos} \\
          \textit{(W): Capítulos 15, 16, 18, 19; (O): Capítulos 6, 7.}
    \item \textbf{Comportamiento en Contextos de Incertidumbre} \\
          \textit{(W): Capítulos 26, 28, 29; (O): Capítulo 9.}
    \item \textbf{Juegos Repetidos y Reputación} \\
          \textit{(W): Capítulos 22 y 23; (O): Capítulos 14 y 15.}
\end{enumerate}

\section*{Notas del profesor}
% \addcontentsline{toc}{section}{Notas del profesor}

Puedes acceder a las notas del profesor en \href{www.google.com}{www.google.com}.

\chapter{Ejemplos e Introducción}